\documentclass{article}
\usepackage{amsmath}



\begin{document}

\section{ Field Variables }
\begin{tabular}{|c|c|c|}
\hline
Symbol & SILO variable name & description \\
\hline
$\rho_i$ & rho\_1, rho\_2, rho\_3, etc. & ith density fraction. \\
$s_i$ & sx, sy, sz & inertial frame momentum \\
$z_i$ & zx, zy, zz & inertial frame spin momentum relative to cell center \\
$g_i$ & gx, gy, gz & gravitational acceleration \\
$\phi$ & phi & gravitational potential \\
$E$ & egas & kinetic + internal energy density \\
$\tau$ & tau & entropy tracer \\
\hline
\end{tabular}

\section{ Other Variable}




\begin{tabular}{|c|c|c|}
\hline
Symbol & SILO variable name & description \\
\hline
$\mu_i$ & atomic\_mass[i-1]  & atomic mass number of ith fraction \\ 
$Z_i$ & atomic\_number[i-1]  & electron number of ith fraction \\ 
$X$ & X[i-1] & hydrogen mass fraction of each species \\ 
$Z$ & Z[i-1] & metallicity of each species \\ 
$\Omega$ & omega & rotation frequency of the grid \\
\hline
\end{tabular}

\begin{tabular}{|c|c|}
\hline
version           & version number for this SILO file \\
code\_to\_g       & units used to convert code mass units to grams \\
code\_to\_s       & units used to convert code time units to seconds\\
code\_to\_cm      & units used to convert code length units to centimeters \\
n\_speces         & number of species, rho\_1, rho\_2, etc. \\
eos               & equation of state 0 = ideal 1 = Segretain \\
gravity           & true if gravity module is enabled \\
hydro             & true if hydro module is enabled \\
radiation         & true if radiation module is enabled \\
output\_frequency & time between outputs - for binary systems this is in \\
                  & units of orbital periods \\
problem           & problem specification (sod, dwd, etc. ) \\
refinment\_floor  & refinemnt density floor in code units \\
cgs\_time         & physical time of output from t=0 in seconds \\
rotational\_time  & time in number of orbits \\
xscale            & the length scale of the grid \\
hostname          & name of the machine this file was produced on \\
node\_list        & list of all nodes \\
node\_positions   & position of each node in the space filling curve \\
node\_count       & total number of nodes in octree \\
leaf\_count       & total number of SILO leaves, this may be less than the \\
                  & total number of octree leaves if compression is on \\
timestamp         & time the file was written \\
epoch             & starts at 0 and is incremented by one for each restart \\ 
                  & since t = 0 \\
locality\_count   & number of localities \\
thread\_count     & total number of threads \\
step\_count       & total number of steps \\
time\_elapsed     & time in seconds from either startup or the last output \\
                  & to the current output \\
steps\_elapsed    & number of steps from either startup or the last output \\
                  & to the current output  \\
\hline
\end{tabular}

\section{Derived Expressions}

total mass density
\begin{equation}
\rho := \Sigma_{i=1}^N \rho_i 
\end{equation}

x - velocity relative to grid
\begin{equation}
v_x  := \frac{s_x}{\rho} + y \Omega 
\end{equation}

y - velocity relative to grid
\begin{equation}
v_y  := \frac{s_y}{\rho} - x \Omega 
\end{equation}

z - velocity relative to grid
\begin{equation}
v_z  := \frac{s_z}{\rho} 
\end{equation}

\begin{equation}
\end{equation}

internal gas energy density
\begin{equation}
s^2  :=  {s_x}^2 + {s_y}^2 + {s_z}^2
e    :=  \begin{cases}
       E - \frac{1}{2} \frac{s^2}{\rho} & \text{if } E - \frac{1}{2} \frac{s^2}{\rho} > 0.001 E \\
       \tau^\gamma & \text{else} \\
     \end{cases} 
\end{equation}

number density of ith fraction (ions + electrons)
\begin{equation}
n_i := \frac{\rho_i}{\mu_i m_H} \left ( 1 + Z_i \right ) 
\end{equation}

\begin{equation}
\gamma := \frac{5}{3}
\end{equation}

total number density
\begin{equation}
n := \Sigma_{i=1}^N  n_i
\end{equation}

temperature
\begin{equation}
T := \frac{1}{\gamma-1} \frac{e}{n}
\end{equation}

pressure
\begin{equation}
P := \left(\gamma-1\right) e
\end{equation}


\end{document}
